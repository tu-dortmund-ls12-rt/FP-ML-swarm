\documentclass[bibliography=totoc, captions=tableheading, titlepage=firstiscover]{scrartcl} %Literatur im Inhaltsverzeichnis, Tabellenüberschrift, erste Seite Titelseite

% Zahlen und Einheiten

%\usepackage[version=4, math-greek=default, text-greek=default]{mhchem}% chemische Formeln
\usepackage{xfrac}% schöne Brüche im Text
\usepackage{scrhack} % Paket float verbessern
\usepackage[aux]{rerunfilecheck} % Warnung, falls nochmal kompiliert werden muss

\usepackage{amsmath} % unverzichtbare Mathe-Befehle
\usepackage{mathtools} % Erweiterungen für amsmath
\usepackage{amssymb} % viele Mathe-Symbole
\usepackage{fontspec} % Fonteinstellungen
% Latin Modern Fonts werden automatisch geladen 
% Alternativ zum Beispiel:
%\setromanfont{Libertinus Serif}
% \setsansfont{Libertinus Sans}
% \setmonofont{Libertinus Mono}

% Wenn man andere Schriftarten gesetzt hat, sollte man das Seiten-Layout neu berechnen lassen
\recalctypearea{}

\usepackage[math-style=ISO, bold-style=ISO, sans-style=italic, nabla=upright, partial=upright, warnings-off={mathtools-colon, mathtools-overbracket}]{unicode-math}

% traditionelle Fonts für Mathematik
\setmathfont{Latin Modern Math}
% Alternativ zum Beispiel:
%\setmathfont{Libertinus Math}

\setmathfont{XITS Math}[range={scr, bfscr}]
\setmathfont{XITS Math}[range={cal, bfcal}, StylisticSet=1]

\usepackage[autostyle]{csquotes} % richtige Anführungszeichen
\usepackage{float} % Standardplatzierung für Floats einstellen
\floatplacement{figure}{htbp}
\floatplacement{table}{htbp}
\usepackage[section, below]{placeins} % Floats innerhalb einer Section halten
\usepackage{pdflscape} % Seite drehen für breite Tabellen: landscape Umgebung
\usepackage[labelfont=bf, font=small, width=0.9\textwidth]{caption} % Captions schöner machen.
\usepackage{subcaption} % subfigure, subtable, subref
\usepackage{graphicx} % Grafiken können eingebunden werde
\usepackage{booktabs} % schöne Tabellen
\usepackage{microtype} % Verbesserungen am Schriftbild
\usepackage[backend=biber]{biblatex} % Literaturverzeichnis
\addbibresource{sample.bib} % Quellendatenbank
%\addbibresource{programme.bib}
\usepackage[german, unicode, pdfusetitle, pdfcreator={}, pdfproducer={}]{hyperref}
\usepackage{bookmark} % erweiterte Bookmarks im PDF
\usepackage[shortcuts]{extdash} % Trennung von Wörtern mit Strichen
\usepackage{url}